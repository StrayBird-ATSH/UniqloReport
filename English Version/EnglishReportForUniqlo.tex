\PassOptionsToPackage{unicode=true}{hyperref} % options for packages loaded elsewhere
\PassOptionsToPackage{hyphens}{url}
%
\documentclass[12pt,]{article}
\usepackage{lmodern}
\usepackage{amssymb,amsmath}
\usepackage{ifxetex,ifluatex}
\usepackage{fixltx2e} % provides \textsubscript
\ifnum 0\ifxetex 1\fi\ifluatex 1\fi=0 % if pdftex
  \usepackage[T1]{fontenc}
  \usepackage[utf8]{inputenc}
  \usepackage{textcomp} % provides euro and other symbols
\else % if luatex or xelatex
  \usepackage{unicode-math}
  \defaultfontfeatures{Ligatures=TeX,Scale=MatchLowercase}
\fi
% use upquote if available, for straight quotes in verbatim environments
\IfFileExists{upquote.sty}{\usepackage{upquote}}{}
% use microtype if available
\IfFileExists{microtype.sty}{%
\usepackage[]{microtype}
\UseMicrotypeSet[protrusion]{basicmath} % disable protrusion for tt fonts
}{}
\IfFileExists{parskip.sty}{%
\usepackage{parskip}
}{% else
\setlength{\parindent}{0pt}
\setlength{\parskip}{6pt plus 2pt minus 1pt}
}
\usepackage{hyperref}
\hypersetup{
            pdftitle={ A Comprehensive Research of the Supply Chain of Uniqlo},
            pdfauthor={Chen Wang},
            pdfborder={0 0 0},
            breaklinks=true}
\urlstyle{same}  % don't use monospace font for urls
\usepackage[margin=1in]{geometry}
\usepackage{graphicx,grffile}
\makeatletter
\def\maxwidth{\ifdim\Gin@nat@width>\linewidth\linewidth\else\Gin@nat@width\fi}
\def\maxheight{\ifdim\Gin@nat@height>\textheight\textheight\else\Gin@nat@height\fi}
\makeatother
% Scale images if necessary, so that they will not overflow the page
% margins by default, and it is still possible to overwrite the defaults
% using explicit options in \includegraphics[width, height, ...]{}
\setkeys{Gin}{width=\maxwidth,height=\maxheight,keepaspectratio}
\setlength{\emergencystretch}{3em}  % prevent overfull lines
\providecommand{\tightlist}{%
  \setlength{\itemsep}{0pt}\setlength{\parskip}{0pt}}
\setcounter{secnumdepth}{5}
% Redefines (sub)paragraphs to behave more like sections
\ifx\paragraph\undefined\else
\let\oldparagraph\paragraph
\renewcommand{\paragraph}[1]{\oldparagraph{#1}\mbox{}}
\fi
\ifx\subparagraph\undefined\else
\let\oldsubparagraph\subparagraph
\renewcommand{\subparagraph}[1]{\oldsubparagraph{#1}\mbox{}}
\fi

% set default figure placement to htbp
\makeatletter
\def\fps@figure{htbp}
\makeatother

\usepackage{etoolbox}
\makeatletter
\providecommand{\subtitle}[1]{% add subtitle to \maketitle
  \apptocmd{\@title}{\par {\large #1 \par}}{}{}
}
\makeatother

\title{\vspace{3in} A Comprehensive Research of the Supply Chain of Uniqlo}
\providecommand{\subtitle}[1]{}
\subtitle{A Report for Uniqlo Scholarship Winter Program (Supply Chain Reform
Direction)}
\author{Chen Wang\footnote{Undergraduate in Software Engineering, Software
  School of Fudan University; Software Development Engineer at Amazon
  Shanghai Institute of Artificial Intelligence, Amazon Web Services.
  (\href{mailto:wangc16@fudan.edu.cn}{\nolinkurl{wangc16@fudan.edu.cn}};
  \href{mailto:cwanam@amazon.com}{\nolinkurl{cwanam@amazon.com}})}}
\date{February 11th, 2020}

\begin{document}
\maketitle

\newpage

\LARGE

\begin{center}
\textbf{A Comprehensive Research of the Supply Chain of Uniqlo}
\end{center}

\large
\begin{center}
\textbf{\emph{A Report for Uniqlo Scholarship Winter Program (Supply Chain Reform Direction)}}
\end{center}

\hypertarget{abstract}{%
\section*{Abstract}\label{abstract}}
\addcontentsline{toc}{section}{Abstract}

With the improvement of living standards, the demand of consumers for
new clothing has become ``fast fashion''. Both clothing varieties and
new styles require apparel companies to respond quickly to market
demands. However, Chinese companies' apparel industry supply chains are
facing slow problems. Since 2002, many international fast fashion brands
have begun to enter the Chinese apparel market, and these fast fashion
apparel brands have taken a unique marketing strategy and brought
different experiences to their customers, which is putting our apparel
brand at a disadvantage. As an excellent fast fashion apparel brand,
Uniqlo, which is the largest fast fashion brand in China, has made great
progress in China, achieving considerable economic benefits and relying
on unique marketing strategies.

One of the major advantages that differentiate Uniqlo from other apparel
companies is its proficient supply chain management. and the quick
response strategy is also an important supply chain management strategy.
In this essay, after a comprehensive analysis of the Uniqlo company, we
come to the conclusion that an excellent supply chain can bring vitality
to the company and make the entire company dynamic. I have also given
some personal insights about what other companies can learn from the
strategies which are adopted by Uniqlo and how these strategies can be
enhanced to have a better result.

\hypertarget{keywords}{%
\section*{Keywords}\label{keywords}}
\addcontentsline{toc}{section}{Keywords}

Supply Chain Management; Quick Response; Uniqlo

\normalsize

\newpage

\tableofcontents

\newpage

\hypertarget{introduction}{%
\section{Introduction}\label{introduction}}

With the economic globalization, production and design competition has
intensified and enterprise supply chain competition has gradually
shifted. Apparel industry is a traditional industry with short cycle and
many characteristics of demand. In the face of supply and demand
uncertainty and high inventory, apparel enterprise customers are
demanding on diversification and market response speed, while they also
want market demand forecast to be accurate. The rapid change of modern
garment industry has shortened the order time, resulting in more and
more product varieties, overstocking and other problems, which has
brought great pressure to the supply chain management of China's garment
industry. At the same time, economic growth and rising living standards
make customer demand very diverse and personalized, which is
increasingly complex and unpredictable. In 2014, China's 87 garment and
textile enterprises had 73.2 billion inventories, 83\% of which were
small and medium-sized garment enterprises. The situation is serious due
to overstocking and a sharp decline in orders and a shortage of sales
channels. It will be very realistic to establish an efficient supply
chain for customer demand, for the purpose of avoiding risks and
improving inventory.

Uniqlo, a business of Fast Retailing Company, will serve as a case study
to answer these questions. Fast Retailing is a specialty retailer of
private label apparel (SPA). It is engaged in the design, production and
sale of clothing. Fast retailing has two advantages. Most of the
``global birth'' companies are high-tech industries. Fast Retailing, as
a clothing retailer, is a low-tech industry. By researching low-tech
industries, we can contribute to the research of ``global'' enterprises.
Fast retailing is not manufactured in Japan. They are produced abroad
without the advantage of domestic production. What's more, as the main
part of this research, we will dig deeper into the supply chain of
Uniqlo.

This essay is divided into seven chapters. The first chapter,
Introduction, gives an overall description of the topics involved in
this essay. The second chapter, The Concepts of Supply Chain and
Specialty Store Retailer of Private Label Apparel, introduces the
generic concepts that will be discussed more specifically later in this
essay. The third chapter, The Growth of Uniqlo Business of Fast
Retailing Co.~Ltd., gives generic description of the development process
and the current situation of Uniqlo company. The fourth chapter, General
Analysis of Uniqlo of the Supply Chain of Uniqlo, analyzes the supply
chain model of Uniqlo company. To make analysis more specific, the fifth
chapter, Comparative Case Analysis Between Uniqlo and Giordano, gives a
comparative analysis between Uniqlo and another typical company in the
aspects of supply chain. The sixth chapter, Analysis of Quick Response
Based Supply Chain Management in Uniqlo, selects a specific strategy as
the interest point to present a more in-depth analysis of the supply
chain model of Uniqlo. The seventh chapter, Conclusion, gives a brief
conclusion of the characteristics of the supply chain model of Uniqlo
and indicates what we can learn from the current supply chain model of
Uniqlo.

\hypertarget{the-concepts-of-supply-chain-and-specialty-store-retailer-of-private-label-apparel}{%
\section{The Concepts of Supply Chain and Specialty Store Retailer of
Private Label
Apparel}\label{the-concepts-of-supply-chain-and-specialty-store-retailer-of-private-label-apparel}}

In this chapter, the specialty retailers of private label apparel (SPA),
which is a form of chain store and supply chain system (SCS) chain
store, are the focus of our research. We will explore the approaches
that ``born-global'' companies use to take advantage of overseas
markets. The Fast Retailing Co., Ltd.~is used as a case. The advantage
of a competitive SPA lies in the ability of the organization to operate
the SCS, in that it first summarizes the existing research on the SCS,
necessitates a strategic perspective, and then the SCS is characterized
by the benefits of SPA.

\hypertarget{existing-research-on-scs-and-strategic-perspective}{%
\subsection{Existing research on SCS and strategic
perspective}\label{existing-research-on-scs-and-strategic-perspective}}

The supply method known as the ``just-in-time'' system, and the parts
procurement network used by Toyota, allow companies to develop their own
SCS. Of particular importance to the rules of thumb used by Toyota is
that ``the company result depends on the quality of operation that uses
external management resources, including the out-sourcing business.''

From the viewpoint of lean operation, Cox (Cox, 1999, pp.167-168) has 8
features:

\begin{enumerate}
\def\labelenumi{(\arabic{enumi})}
\item
  We aim for perfection to provide value to customers.
\item
  Only for actions that produce what is extracted from customers
  just-in-time and create a focused value flow.
\item
  Focus on waste removal in all in-house operational processes and
  externally, overproduction, standby, transportation, improper
  processing, defects, unnecessary inventory and movement.
\item
  All participants in the supply chain are stakeholders and need to add
  value to everyone in the business.
\item
  Closer, collaborative, reciprocal, not trustworthy (win-win), arm
  length and adversary (win or lose), supplier relationships.
\item
  Work with suppliers to create a lean, demand-driven logistics process.
\item
  A desirable long-term relationship that reduces the number of
  suppliers and works more intensively with a given supplier.
\item
  Create a network of suppliers and build a common understanding and
  learning products and services on waste reduction and operational
  efficiency in existing deliveries.
\end{enumerate}

In general, Cox's research is based on a company and its suppliers.
Later, he defined SCS as the following pattern of operations: Avoid
waste and add value to your customers. Cox introduces a strategy-level
concept that highlights the unique nature of the operational supply
chain (Cox, 1999, p.168). He underestimated the importance of vertical
relationships between companies that systematically criticized existing
strategic research. He referred to the concept of core competence as
existing work. The success of the technology and resources business
owned and managed by companies that regard the concept of cockscore
competence as an internally defined concept. Therefore, Cox could not
explain the supply chain in which the core conceptual capabilities
existed between some companies beyond the boundaries of the company.
Finally, he has strategic management. In other words, Cox discussed SCS
at the strategic level.

\hypertarget{characteristics-of-chain-store-scs}{%
\subsection{Characteristics of chain store
SCS}\label{characteristics-of-chain-store-scs}}

``In the case of a chain store, each product is just a part, a
material.'' This description is a description based recognition for
chain store SCS.

A manufacturer's end product is made up of many parts and materials.
Therefore, customer satisfaction is achieved by connecting the
procurement system parts and materials in the production system. But
from a chain store perspective, a manufacturer's finished product is
simply an intermediate in the parts, materials, and marketing process.
The finished product chain store is a variety of products that customers
demand. For this reason, assortment is an important word. The meaning of
``assortment'' is defined as a result of commercialization by retailers.
That is, to achieve customer satisfaction, retailers need to determine
the goods they sell, and the price and quality of the goods. It is
determined by this choice as it appears in the merchant retail store. We
have selected this product group as assortment.

In summary, the chain store's production system is complete. Chain store
results by the quality of the procurement system. Manufacturers examine
the quality and price of product parts and materials while avoiding
waste. At the same time, as a manufacturer, chain stores examine quality
and price. Each product functions as a part and material and wants to
avoid waste. This means that unsold products are removed from the shop
floor. The tendency to avoid waste in this chain store is that the
products that customers demand are perfect. Customer demand by providing
a complete assortment of goods that chain stores meet.

Usually, the sales system is a production system for retailers.
Therefore, sales capacity is considered a criterion for judging the
results of the retailer. However, adaptation and modification of the
procurement system to the production system, merchandise procurement
system, and chain store chain store sales system. For example, some
mottos are ``high quality, low price'', ``daily low price'', etc., sales
system. Rather, the motto requires a customer. This is the result of the
sourcing system starting from trying to meet the requirements of the
client. These mottoes reflect the assortment that the chain store
offers, or tries to offer.

The main activity of SCS of the chain store is ``assortment'' (Yoshida,
2001). This edition is an activity that retailers will consider a
combination of products assorted in shop flowers. In other words,
edition is an activity that retailers consider an assortment to meet
customer satisfaction. Therefore, the edition activity is to resolve any
discrepancies at the end of the marketing process, creating and
delivering customer satisfaction. In the other words, to completely
remove the discrepancy on the final stage of the marketing process means
the creation and offering of the customer satisfaction. The assortment
offered by the retailer makes it possible to remove the discrepancy to
the customers. Therefore, assortment is a starting point for the
marketing process.

To put it concretely, to completely eliminate the divergence that the
chain of stores intends is embodied by the assortment presented to end
customers. The making of this assortment depends on the editing activity
of the assortment. Therefore, editing the assortment is a ground-based
design for realizing the actual chain store assortment, and also a
strategic subject. If the chain of stores cannot match the goods along
the edition of the assortment, the edition does not make sense. Namely,
SCS that the activity to really supply the product is a main role is of
strategic importance. Thus, the quality of the ability to organize the
SCS decides the company's competitive advantages. Therefore, SCS is not
just a supply chain for a product. Rather, it is a type of
organizational behavior that can eliminate the gaps that may arise in
the final stages of the marketing process. And the ability to organize
organizational behavior becomes the specific advantage of the company.

\hypertarget{spas-competitive-advantages}{%
\subsection{SPA's competitive
advantages}\label{spas-competitive-advantages}}

Chain prices alone have no competitive advantage at low prices. It's
easy to use low prices as a basic tactic, but it's just one aspect of
SCS procurement activities. In short, this tactic arises from non-price
competition. Therefore, after defining the feasible range of the
assortment, the practical ability to procure within that range is the
ability to organize SCS (Yoshida, 2001).

The smaller the feasible range of the assortment, the narrower the scope
of the procurement activities of the product: Therefore, solutions that
require removal of ranges and discrepancies are deepened. Conversely,
the greater the feasible range of assortment, the lower the density of
product procurement activities. The easy-to-use chain has 50,000 items
that image the qualitative differences between the activities stored,
and the assortment of goods and the chain store with only 10 items edit
the assortment. It is a general store (GMS) chain store, SCS. GMS
overcomes the strategic subject of editing the assortment in the store
floor, and it is impossible to achieve an assortment of goods along the
edition. This indicates that GMS intends a high assortment of customers
that could cause inconsistencies with chain stores.

Chain stores are trying to reveal their unique competitive advantage by
infusing each product with its own strategic intent. In other words, if
the chain store establishes a competitive advantage, it will edit the
assortment by removing the inconsistency between the store and the
customer, and the goods in line with this edition that will practice the
actual assortment behavior, The intentions that make will be natural The
unique strategic intentions will penetrate to each stage of production
functioning goods. This will affect the assortment, even if the product
is a national brand. For example, vegetables are cultivated, production
is not only aimed at securing the supply of vegetables, but with large
seasonal price fluctuations. The contract is part of a procurement
activity along a proprietary edition of the assortment. That is, the
contract contains not only a negative intent to secure supply, but also
an aggressive intent to realize an assortment of editions. This is an
important element of SCS. Also, if a national branded product is
inconsistent or does not fit its own edition of the assortment, it is
natural that the product has been destroyed from the assortment. As a
result, private brands are created by chain stores.

SPA is a type of chain store that specializes in clothing, and a company
that edits its own assortment using private brands. The private brand
assortment is the key to SPA achieving its unique strategy. SPA will
consider an assortment of editions using a private brand assortment,
building commodities, and security SCS to source commodities according
to their own intent. This gives SPA an advantage over other chain
stores.

\hypertarget{the-growth-of-uniqlo-business-of-fast-retailing-co.ltd.}{%
\section{The Growth of Uniqlo Business of Fast Retailing
Co.~Ltd.}\label{the-growth-of-uniqlo-business-of-fast-retailing-co.ltd.}}

This chapter describes the Fast Retailing (FR) Uniqlo business. First,
an overview. Second, the history of Fast Retailing is shown. Third, I
will discuss the ``all better change'' (ABC) activities that have had a
significant impact on FR's management style. Finally, the current status
of FR is displayed.

\hypertarget{overview-of-fast-retailings-uniqlo-business}{%
\subsection{Overview of Fast Retailing's UNIQLO
business}\label{overview-of-fast-retailings-uniqlo-business}}

FR is responsible for the entire business process from planning to
sales. The casual wear product brand is UNIQLO. FR operates a UNIQLO
chain store.

UNIQLO is a casual wear brand and the name of a chain owned by FR. FR
considers casual wear widely. The company does not sell formal wear,
such as swallow tail coats or evening dresses. We sell underwear, belts,
caps, bags and other products. In this sense, the range of items that FR
sells is casual wear and the items needed to wear casual wear.

FR's total sales and stores have grown significantly since it opened in
1984. Total sales in 2001 were 418,500,000,000 yen. There are 519
stores. Ordinary profit was 103,200,000,000 yen and the ordinary profit
ratio was 24.7\%. In 2002, gross sales are expected to decline due to
rapid growth in 2001. Total sales in 2002 were 204,800,000,000 yen. The
expected ordinary profit is 38,500,000,000 yen. The estimated number of
shops is 555. FR has recorded sales and profits for 11 consecutive
years. The shares were listed on the second section of the Tokyo Stock
Exchange in 1997 and on the first section of the Tokyo Stock Exchange in
1999 (Togawa et al., 2000, P.26). In August 2001, the company employed
1,598 full-time and 10,674 part-time workers.

FR's corporate philosophy and management principles were developed by
Tadashi Yanai, the founder of FR. Much of this philosophy involves
universal and simple content, applies to every company in every
industry, and then internalizes the founder's belief in ``naturally
customs'' (Togawa et al., 2000, p.~127)., ``FR lists the following three
points as key guidelines for the Uniqlo business.''

\begin{enumerate}
\def\labelenumi{(\arabic{enumi})}
\tightlist
\item
  Always improve the product
\item
  Strengthen low-cost operations
\item
  First to place customers
\end{enumerate}

To prioritize customers, Uniqlo maintains a clean shop and full
inventory, and allows returns for three months from the date of
purchase. Through strictly maintaining these principles of the UNIQLO
business, FR plays a key role in managing that customer satisfaction is
at its highest position, playing with the belief that customers can
achieve their management.

The operational characteristics of FR are implied by company name. First
of all, fast means instantly generating the demands that customers have
on a commercial basis. The phrase also reflects the founder's strong
desire to become a retailer with a ``fast food concept''. The idea is,
some restaurants, such as McDonald's was to transfer the concept of de
company in the clothing industry. First of all, fast food can be eaten
anytime, anywhere. Similarly, UNIQLO branded garments are produced with
the goal of allowing anyone to wear them anytime, anywhere. FR products
are popular and basic because they target ``anybody, anywhere,
anytime''.

Low prices are affordable products for everyone. In addition, FR is sold
in unisex style for all ages. Second, fast food stores offer the same
items and services all over the world. UNIQLO, a FR shop, aims to
provide the same items and the same services at all shops throughout
Japan. For Uniqlo, each store uses standard layouts and operating
procedures (Chikae, 2000, pp.~110-111). Third, companies that operate
fast food chains have their own systems of planning, development and
sales. Similarly, FR designs, plans and sells its products. FR organizes
production networks in China and Southeast Asia. FR sells its products
in its own sales network, the Uniqlo shop. A fast food restaurant and
the same way, FR for large-scale 1 to produce a single product, offers
at a low price in all stores. This idea is supported by the fact that
inventory is limited to 200 items (Hatano, 2000, p.34). Finally, fast
food companies prioritize reducing labor costs. FR will build a store of
the same size as possible (approximately 495 square meters) and place it
in the suburbs (Wol, 2000, p.~25). The name UNIQLO is meant to imply ``a
unique clothing warehouse.'' UNIQLO is a store similar to a warehouse.
This style prevents waste. Uniqlo uses a ``help yourself'' system
modeled after a supermarket. The cost of starting a new store is between
60,000,000 and 70,000,000 yen (Chikae, 2000, p.110). FR practices
labor-saving and low-cost construction.

\hypertarget{establishment-of-uniqlo}{%
\subsection{Establishment of UNIQLO}\label{establishment-of-uniqlo}}

FR started with men's clothing store Ogori Trading. The company, 1949 in
Ube City, Yamaguchi Prefecture in years, FR of the CEO was founded by
Hitoshi Yanai is Tadashi Yanai of the father is. Shoji Ogori was run by
Hitoshi Yanai. The main seller was high-quality formal wear for
gentlemen (Okamoto, 2000, p.87). The company purchased goods from Gifu
and Nagoya, which are involved in the textile industry. The current FR
of CEO is a positive Yanai is a 1972 joined years. He graduated from
Waseda University in 1971 and later worked at JASCO, a major Japanese
chain. Ogori Shoji sold well-known Japanese casual wear brands,
including formal wear for gentlemen, foreign brands of high-quality
women's dresses, and VAN. In 1984, when Tadashi Yanai assumed office as
a director of Ogori Shoji, he established UNIQLO. The first store opened
in Hiroshima near Ube City. The bubble economy started around 1985. At
this time, expensive designers and ``character brand'' fashion were
popular in Japan. Under these circumstances, UNIQLO started its business
using the concept of selling casual wear at low prices. The concept is
represented as follows: ``In the store you can buy clothes like a
magazine.'' Shop 10 decided to target a boy of generations. At this
time, the Uniqlo concept has not yet been fully developed. The store
inventory did not include the Uniqlo brand. However, it was
characterized by low prices. The ``help-yourself'' system has been
adopted. This system was started by Yanai who believed that high-quality
shops must serve their customers, but that casual wear customers can
help themselves. In 1984, Ogori Syoji became the chairman of Chuchu
Yanai.

In June 1985, UNIQLO opened on the outskirts of Shimonoseki. 10 May,
similar Uniqlo shops in Okayama 2 was hotels open. The store was built
to look like a warehouse to save construction costs. This style has
saved overhead. Most of the Uniqlo inventory has been imported. In 1985,
the yen exchange rate increased significantly in Japan. Still, the cost
of goods purchased by FR did not decrease. Therefore, FR needed to
reduce overhead costs to achieve lower prices at the point of sale. In
1988, the franchisee was adopted to reduce costs by purchasing large
quantities.

On the other hand, the opening of new stores has made it difficult to
steadily find sources of high-quality, low-priced products.

In 1987, Yanai went on a Hong Kong tour, visiting a company called
Jelldarno, which had a brand name of the same name. This brand was sold
in the United States and Europe. The company had a SPA format. This
brand has a reputation for high quality and low price. Yanai has decided
to outsource the production of Uniqlo branded items to a factory that
manufactures Jelldarno products. This was the first step towards
developing a SPA that controls planning, production, distribution and
sales. This system is modeled on GAP. In 1988, the Point of Sale (POS)
system was introduced. POS systems use computers to quickly track
inventory in stores and make that information available to headquarters.
With the introduction of this system, management from planning to sales
has become possible. In the same year, FR began full-scale development
of the Uniqlo brand (Okamoto, 2000, p.111). In February 1989, FR
established an Osaka office to enhance product development. The office
involved a supplier in the development of a specialty product. In 1989,
FR also built a distribution center. In 1990, the computer system was
updated to handle internal merchandise and sales information. In 1991,
the company name was changed from Ogori Shoji to Fast Retailing. Total
sales were 7,179,000,000 yen and there were 29 stores.

In 1992, Ogori Trading, which specializes in formal wear, was changed to
UNIQLO, and all FR shops became UNIQLO. New computer systems have been
introduced to grow the business and implement management strategies. In
1994, the number of FR shops exceeded 100. In July 1994, FR shares were
listed on the Hiroshima Stock Exchange. List FRs released from financial
issues such as bank loans. FR moved quickly to open more Uniqlo shops.
In 1996 there were more than 200 stores and in 1997 there were 300
stores. Then, FR shares of the Tokyo Stock Exchange 2 was listed on the
second section. On the other hand, to strengthen the Uniqlo brand in, FR
is 1994 years 12 founded the design subsidiary in New York in May. This
subsidiary FR is 100 owned\%, was aimed at enhancement of the design and
information collection. In order to strengthen the production, FR is
1996 founded the production subsidiary in China of Sandton in years. The
subsidiary was a joint venture between five companies, including
Nichimen, a general trading company, and a Chinese company. FR's
investment ratio was 28.7\%.

In 1997, FR launched Spoqlo, a casual sportswear shop and Famqlo, with 9
new types of shopping and shops as children and women as new businesses.
But these businesses were not very successful. Under these
circumstances, the total sales of existing UNIQLO stores were below the
level of the previous year. As a result, the increase in revenue and
profit from all shops was achieved by the sales generated by the
construction of the new shop. 1997 to the year, Sawada Takashi, FR
original COO is, Itochu left Syoji of (a comprehensive company Syosya a)
FR took part in the management of. Sawada, in response to a request of
Yanai, FR in order to improve the sales results of 4 made a single
proposal. He first suggested closing the Spokro and Famcro stores. The
problem was that entering a new business dispersed the vector of
employees. In 1998, FR closed its Spocro and Famcro shops. His 2nd
proposal, was to review the UNIQLO business. Third was to simplify the
business. Until 1997, UNIQLO Shops sold other brands. The brand was sold
at a low price to attract customers. UNIQLO's marketing goals were not
clear to its staff and customers. Therefore, the FR clarified its
purpose. UNIQLO has launched sales of its own brand of casual wear. His
4th proposal, was to close the New York design office. The offices in
Osaka, New York, and Tokyo each had separate design facilities until
1997, but the overall approach was inconsistent. These offices were
merged in 1998 and a new office was established in Tokyo.

In 1998, a new FR board was appointed. FR's management was young and
under the guidance of Yanai and Sawada. The above improvements are
currently being carried over as ``All better change (ABC)'' activities.

\hypertarget{abc-activity}{%
\subsection{ABC activity}\label{abc-activity}}

Yanai is, 1998 years 6 began in May ABC the nature of the activity was
defined as follows: (iro incarnation, 2000 years 1 month, 36 pages).

``So far, headquarters have been thinkers and stores have practiced. We
will change this style to a new style of thinking and practicing in all
workplaces, including shops. Change to a style that meets the needs of
the customer, that is, how to produce marketable products. ABC
activities are literally everything that changes.''

The first part shows that the headquarters will take the initiative and
change the store operation to a store operation that is autonomous by
the store manager. A management manual was used at UNIQLO shops to
standardize operating procedures in Japan. It was good management to
practice the manual in the store. On the other hand, reliance on manual,
or take away the thinking ability of the manager, the store did not use
the idea of the length. Gradually, the negative effects of the manual
exacerbated the FR problem. To change the situation, FR has changed its
organization and HR policies. At this point, the FR manager was not
responsible for the sale. They were evaluated based on cleanliness,
inventory management and human resources management. Since July 1998,
sales have been added to the list of responsibilities. In February 1999,
the Superstar Manager system was introduced. The Superstar Manager was a
full-time FR staff member, whose annual income depended on shop
performance. The system has changed management by linking performance to
rewards. 2001 In the year, 520 in the person of the manager 30 there was
a person of superstar (Weekly Toyo Keizai, 2001-year 11 March 3 days,
P.34). In addition, a supervisory position was created and promoted.
Supervisors were responsible for several shops in defined areas and
served as sales coaches. A supervisor location was set up to identify
problem areas within the store and work with headquarters and store
staff to resolve the problem.

Later in the statement on ABC's activities, Yanai mentioned the shift
from selling products to manufacturing marketable products. This change
is a matter of restructuring the supply chain system (SCS). FR's design
office has been integrated into Tokyo, strengthening the unification of
the UNIQLO brand. FR narrowed down the items from 200 to 300. FR,
instead of increasing the number of items, in order to reduce the cost 1
was creating a lot of one item.

FR is to introduce a system with a sales strategy with the weekly demand
forecast, 1998 years 10 has been changed a production plan to match the
sales of the May plan. In 1998, 90 \% of FR products came from factories
that contracted production in China. FR has reduced the number of
factories from 140 to 40 (Guisen, 2000, p.24). FR is 2001 in year 85 of
the plant (60 commissioned the production of clothing company). Most
Japanese SPA companies have consigned production contracts with
factories in China via general trading companies. SPA companies have
turned production management and quality control into the hands of a
general trading company. In opposition to this position, FR is 1999
years 4 a Shanghai office in May, the same year 9 founded the Guangzhou
office in May. The China office has 90 local staff members who visit the
factory every Tuesday to Thursday to work directly on quality control.
Also, since 2001, FR has organized a skilled team called Takumi. The
team members are experienced Japanese engineers. They are a group of
people who have worked for a Japanese textile company for a long time
and have experience as factory managers.

This team consists of 14 people. In one month, one person visited 10
factories in China and transferred factory management and sewing
technology. In addition, in May 2001, FR established an online system
between these Chinese contract factories and the Japanese headquarters.
Factory production data is sent online to headquarters. In addition,
orders can be immediately delivered to the factory based on sales data.
FR has established a production system that produces 50\% of the
production plan at the beginning of the season and produces the
remaining products according to the sales volume in stores. FR shares
production information with several material companies that are
responsible for the production of cotton, yarn, textiles and dyes. FR
links production and sourcing of materials.

To manage distribution, FR staff reviews products delivered to a Chinese
trading warehouse. FR has distribution centers in Tokyo and Osaka.
Distribution centers classify primary products by size and color.
Products are delivered to the shop three times a week. Delivery is to an
external company.

In November 1998, FR opened a new shop in Harajuku, Tokyo. FR has
changed its advertising strategy from using house-to-house flyers to a
multimedia approach that uses newspapers, magazines, and television. As
a result, the Harajuku store was successful and the UNIQLO brand was
strengthened. At the same time, the possibility of developing a store
outside the suburbs was shown. In addition, FR was listed on the first
section of the Tokyo Stock Exchange in November 1999. A new computer
system, introduced in October 2000, allowed inventory control by color
and size with sales demand defined as sales information management.
Minimum number of units. By introducing this system, it is possible to
practice adjustment for production in a smaller unit than the current
unit. In 1999, some shops could be ordered directly from the factory
instead of relying on headquarters. By 2001, 100 out of 520 stores could
be ordered directly in items, colors and sizes. This system will be
introduced in all shops in 2002.

\hypertarget{new-subjects}{%
\subsection{New subjects}\label{new-subjects}}

ABC activities continue today. ABC activities meetings are held every
Monday morning (about 70 people in total) by managers such as section
managers. At these meetings, we were able to discuss company issues.
These meetings and the sales meeting held on Monday afternoon were
important for making decisions about the normal operation of the FR. The
sales meeting was used to discuss sales between sales representatives,
supervisors, and store managers. The results of ABC activities have had
a significant impact on the establishment of FR business processes.

FR is involved in too many projects. FR continues to make progress after
establishing business processes. In 1999, FR launched a mail order
business using brochures. In January 2000, FR collaborated with Simree
in the mail order business. Simree is a company with mail order
experience. In October, a mail-order system via the Internet was
launched. In addition, some existing suburban stores have been closed,
and FR is building new luxury stores.

In June 2000, Fast Retailing (UK), Ltd.~was established as a preparatory
stage for developing UNIQLO overseas. In September 2001, four UNIQLO
shops opened. The locations were Knightsbridge, Wimbledon, Axbridge, and
Romford. The conditions for finding a store were different from those
used in Japan. Knightsbridge is a downtown shop. The Wimbledon shop is
on High Street in this town. Uxbridge and Rom ford shops are located in
suburban shopping malls with large parking lots. Due to legal
differences between Japan and the UK, it was not possible to use
locations along roads outside the UK. Building space is limited, so
finding a store for the FR is an important issue. UNIQLO inventory is
generated in China. The UK has a quota for imports from China. Thus, as
the number of shops grows, maintaining the source of the product becomes
an issue. It emphasizes the importance of partnerships that form
operational and production contracts. FR is, in order to achieve the
product quality is desire, looking for a partner that can build a
cooperative relationship. And if such a partner owns a factory in a
country where the UK does not charge quotas, FR must resolve the issue
of suppliers in the UK market. FR is required to establish a supply
mechanism that emphasizes meeting customer demand in the UK. In
addition, in order to enter the Chinese casual wear market, 2001 years 8
to China in May Fast Retailing (Jiangsu) Apparel Co.~to determine the
success of the future in overseas markets.

\hypertarget{general-analysis-of-uniqlo-of-the-supply-chain-of-uniqlo}{%
\section{General Analysis of Uniqlo of the Supply Chain of
Uniqlo}\label{general-analysis-of-uniqlo-of-the-supply-chain-of-uniqlo}}

The point of our discussion is to consider the meaning of a company
starting an international business without advantages. FR is analyzed
from two perspectives. First, I will explain FR's relationship with
international business. This includes considering whether FR is a
``born'' company. Next, we analyze FR's location advantages in China. We
will also consider why other SPAs cannot mimic the FR model. After these
discussions, we analyze the implications of businesses without
advantages engaging in international business.

\hypertarget{is-fr-a-born-global-company}{%
\subsection{Is FR a ``born-global''
company?}\label{is-fr-a-born-global-company}}

We borrow the concept of the definition of ``global company'' from
Oviatt and McDougall (1994). They argued that it was essential for
``global'' companies to have an international origin. In other words,
``global'' companies need to invest resources in multiple countries.
Therefore, the explanation is (Oviatt and McDougall, 1994, p.~49):

``They do not necessarily own foreign assets; foreign direct investment
is not mandatory; they can arrange strategic alliances to use foreign
resources such as manufacturing capabilities and marketing.''

You need to determine if Fast Retailing (FR) is inherently global. The
criterion is whether the company uses foreign resources from the
beginning.

In the case of FR, the origin was 1949. It was Shoji Ogori started by
the father of FR CEO. Ogori began selling formal wear for men. Later,
the company expanded dress and casual wear and sold it to women. In this
sense, Tadashi Yanai joined the company in 1972. When he served as
Managing Director in June 1984, he launched a casual wear shop, UNIQLO.
After UNIQLO was established, Yanai became the chairman of Ogori
September 1984 Shoji decided that he would operate UNIQLO as a business
of Ogori Syoji. Therefore, 1984 can be considered the year of
establishment of UNIQLO in FR.

UNIQLO was initially a competitive clothing store focused on casual
clothing for teenagers at low prices. The competitive strategy was
supported through the procurement of low-priced products. FR inventory
was from foreign countries, including Hong Kong. When prices for foreign
goods did not drop, warehouse-style stores were devised. Since the
beginning of UNIQLO, business relationships with foreign companies have
been a key factor in the competition to make products cheaper than
Japanese companies. In other words, for FR to grow, it had to use
foreign sources to keep prices low. In 1984, the only aspect of the
business managed by FR was sales.

In summary, FR used foreign resources from the beginning. To determine
if this company is a ``global company'', you need to determine whether
it used foreign resources from the beginning. FR did; therefore, we can
conclude that FR was a ``global company''.

\hypertarget{location-advantages-and-organizational-capabilities}{%
\subsection{Location advantages and organizational
capabilities}\label{location-advantages-and-organizational-capabilities}}

FR has adopted SPA foam. The ability to organize a supply chain system
(SCS) has been shown to be a significant advantage for SPA companies.
SCS is a system that realizes an ideal product lineup. To achieve an
assortment, some activities of SCS have to find some countries. The
season of this place is to integrate the advantages of the places that
exist in each country into the SCS. In other words, location benefits
add value to each product. In addition, organizational capabilities
increase the value of the store to the customer and achieve the
company's intended lineup. In summary, the goal of SPA
internationalization is to combine attractiveness of places with
organizational capabilities to form an attractive assortment.

In the case of FR, when UNIQLO was established in 1984, FR management
was limited to sales functions. The only way to turn a foreign location
advantage into a FR advantage itself was to procure products produced by
foreign manufacturers by trading in international markets. Although FR's
strategy was to target teenagers in the casual wear market, the company
had limited ability to get the products its customers wanted. Assortment
editions and sourcing activities are related to each other. Due to the
limited procurement capacity of the goods, the assortment is compiled
from the goods that can be procured. The inability to edit the
assortment as a strategy drives the establishment of a system for
procuring products. In 1987, FR began producing commissions in the
chain, and the number of factories that contacted for commission
production increased. In 1988, FR began to enhance product planning and
design. A design office was established in Osaka in 1989. In 1988, FR
introduced a computer system to manage the sales of individual stores.
As a result, FR was able to control product planning, part of
production, and sales.

When FR procured only foreign brands through market transactions,
location advantages were incorporated into the products purchased.
However, by starting commission production with a partner, the
advantages of the place are factors that determine the attractiveness of
the product. Most of the factories FR uses to manufacture products are
made in China. The next factor is the advantage of using factories in
China (Weekly Toyo Keizai, November 3, 2001, p.~44). First, the labor is
cheap. China's wages are 1 / 30th of Japan's. For the 700 million people
that make up the working population, those working in the agriculture
and fisheries industries move to urban areas to become factory workers
and return to agricultural areas within three years. This means that
labor costs do not increase because of the use of unskilled workers.
Second, China's main industry is the textile industry. In Shanghai, you
can easily source the materials you need to make garments, such as
buttons, zippers, and fabrics. The price of the material is very low. To
integrate the benefits of using a Chinese facility into SCS, it is
important to strengthen cooperation between FR and Chinese factories. In
other words, strengthening relationships with foreign factories will
enhance SCS's organizational capabilities.

After commissioning commenced, FR's targeted assortment of editions
increased the likelihood of being realized by FR's subjective intent
without being affected by market conditions. However, FR could control
very few parts of the entire business. Most products are foreign-made.
However, if each activity in the FR business process is not enhanced, it
makes less sense for the FR to control all business processes. A design
subsidiary, founded in New York in 1994, aims to enhance product
planning. The design subsidiary assumes that FR wanted to integrate the
benefits of location into business processes.

When the design firm was reorganized in 1998, the design subsidiary in
New York was closed. The reasons for reorganizing the design offices in
Osaka (established in 1989), New York, and Tokyo (established in 1996)
have been described earlier. The Uniqlo brand did not have the
uniformity of three independent offices. In other words, the assortment
version that FR doses is not clear as FR strategy. It was the effect of
procurement on products that FR could actually procure. This was the
result of confusing the lineup that FR envisioned as a strategy and the
lineup that FR could actually procure. In the restructuring of the
UNIQLO business that started in 1998, FR decided that the FR should aim
for the assortment of its confirmation, only the UNIQLO brand, and
basically assorted products at UNIQLO stores. Therefore, the business
process of FR was simple.

FR was a real SPA form with a selection of UNIQLO branded products at
the UNIQLO shop. By using SCS and realizing an ideal assortment, it
becomes a real SPA form, and FR has an advantage over other forms of
commerce. In short, FR can ensure an assortment that will satisfy its
customers. On the other hand, the FR has to bear the cost of maintaining
the SCS. For FR, SCS rust means loss of competitiveness. Since 1998, FR
has used the system to forecast customer demand and change its
production plans weekly accordingly. To maintain these benefits, FR
needs to continually improve and invest in SCS.

This is an enhancement of the well-functioning SCS organizational
capabilities. Each part of the SCS has been improved by enhancing the
quality control system at the factory in China and promoting the
transfer of technology by the artisan team. Through these activities,
FR's production activities are taking advantage of the location
advantages of China and increasing the value of the entire SCS. As noted
above, as growth has grown, the FR has expanded the activities they can
control. To achieve the ideal lineup, FR needed SCS to procure products.
FR's SCS has been enhanced with a commission agreement with a Chinese
plant. In order to integrate China's location advantages into
organizational capabilities, FR has practiced many effort devices, such
as implementing a quality management system.

\hypertarget{the-meaning-of-internationalization-without-benefits-internationalization-to-build-benefits}{%
\subsection{The meaning of internationalization without benefits:
internationalization to build
benefits}\label{the-meaning-of-internationalization-without-benefits-internationalization-to-build-benefits}}

Global non-market globalization has no benefits. The answer is given in
the analysis above. When UNIQLO was founded in 1984, sourcing low-priced
products by engaging in business with foreign companies provided FR with
certain benefits. As growth grew, FR expanded the activities it could
control with SCS. At the same time, the capacity level of each SCS
activity has been enhanced. Finally, all activities of the SCS have been
brought under FR control. Behind this was the reason FR had to build an
SCS to achieve the ideal lineup. The SCS needs to be built as a device
that allows the company to procure the necessary products. In other
words, through the procurement process, SCS must have the ability to add
products to the elements that the company wants. Being part of an SCS
activity in a foreign country means that the SCS can add extra factors
to the product by integrating the locational benefits of the country
where the activity is taking place. In the case of FR, FR's SCS was
produced in China, so we can offer high quality products at low prices.
Owning and operating such an SCS is an advantage of FR. As understood in
the case of FR, internationalization of non-global markets is to
integrate location benefits into their own business processes. By
integrating the benefits of location into business processes, companies
can compete with other companies. Building and capturing the benefits of
the business process itself is why non-market born global companies
engage in international business.

\hypertarget{comparative-case-analysis-between-uniqlo-and-giordano}{%
\section{Comparative Case Analysis Between Uniqlo and
Giordano}\label{comparative-case-analysis-between-uniqlo-and-giordano}}

\hypertarget{introduction-of-the-case-analysis}{%
\subsection{Introduction of the Case
Analysis}\label{introduction-of-the-case-analysis}}

While the Asian apparel industry continues to grow in the global market,
the global apparel indus tri becomes very unstable as the world is
internationalized (Lopez \& Fan 2009). One of the notable trends in the
industry is the aggressive international expansion of Asian apparel
brands. The case is two major Asian brands: Giordano in Hong Kong and
Uniqlo in Japan. They attracted early attention by the Western media
compared to other Asian retailers. For example, Giordano's story is
being discussed as an independent chapter in a Western marketing
textbook (Wirtz 2007), and UNIQLO's flagship opening is on Fifth Avenue
in New York City, where U.S. news (e.g., I received a writing in Dickler
2011). Giordano achieved about 70\% of its total overseas sales in the
recent period (Giordano Interim Report 2013), but the company's total
sales now show stability after 25 years of business. Sales of Fast
Retailing (Uniqlo owners) have increased significantly, and UNIQLO
International's sales account for about twice that of UNIQLO JAPAN
(UNIQLO Annual Report 2013). It is clear that these Asian apparel brands
are becoming global now.

Despite Asia's rapid presence in the global market, Asia has simply been
considered a supply chain producer or manufacturer offering cheap labor.
The debate on the internationalization of Asian apparel brands is quite
lacking in literature. Most of the research on the internationalization
of apparel brands is for specific Western apparel brands focused on
marketing strategies such as Levi's and Zara (Bhardwaj et al.~2011;
2005; Lopez \& Fan 2009; These brands may have a higher brand awareness
and a more stable brand position in the global market for
internationalization for Asian brands. How did Asian brands become
internationalized in the global market? What are the specific strategies
that promoted their internationalization? Are these strategies similar
or different to Western competitors that have already been studied?

To answer these questions, the study aims to apply existing
internationalization theories, identify strategies to promote them, and
analyze the internationalization patterns of two pioneering Asian
brands, Giordano and Uniqlo. International. For this reason, the current
study has previous internationalized case studies (e.g., Childs \& Jin
2014; Lopez \& Fan 2009; Vrontis \& Vronti 2004). The results of this
study not only bridge the theoretical gap in internationalization in the
literature of Asian brands, but also provide a useful information
infrastructure for branding practitioners of the internationalization
case of Asian brands.

This study first reviews the internationalization theory as a framework
for internationalization in apparel brand literature and past case
studies. Next, we analyze the internationalization patterns from the
onset of Giordano and Uniqlo, and apply the internationalization theory.
The common strategy of the two brands for internationalization is
identified and contrasted with major competitors such as global fast
fashion brands. Discussions and implications are also included.

\hypertarget{the-comparative-analysis-between-uniqlo-and-giordano-in-the-perspective-of-supply-chain-a-major-strategy-facilitating-both-companys-internationalization}{%
\subsection{The Comparative Analysis between Uniqlo and Giordano in the
Perspective of Supply Chain: A major strategy facilitating both
company's
internationalization}\label{the-comparative-analysis-between-uniqlo-and-giordano-in-the-perspective-of-supply-chain-a-major-strategy-facilitating-both-companys-internationalization}}

Giordano and UNIQLO exhibit an integrated supply chain system that
company plans and manages the production of its products and sells it to
its specialty stores. Since Giordano started as a manufacturer in the
1970s, it has operated its own manufacturing system, inventory system
and specialty store. Currently, about 95\% of its products are
outsourced, but we offer another 5\% of Giordano's product needs
(Kandelwal \& Saxena 2010) and maintain the main manufacturing
operations that control it through Hong Kong's headquarters. The company
operates as a platform to procure from suppliers in mainland China,
South Korea and Singapore, and can price its value by manufacturing most
of its products in China, where labor costs and factory costs are low.
In this way, Giordano combines the advantages of the low-cost regional
environment of the brand's ``value of money'' concept with the
advantages of Hong Kong's just-in-time logistics system and its
outstanding international marketing network.

UNIQLO represents a more integrated supply chain system that encompasses
every stage of the supply chain, from design and production to final
sales to consumers. This system lowers it to ensure high quality
products at a reasonable price. For example, the company has exclusive
partner manufacturing plants such as Kaihara Corporation for denim and
Toray Industry for textiles, and supplies stable quality products to
stores. The brand has its own R\&D center and outsources production of
about 85\% of Chinese products (UNIQLO Annual Report 2010).

Previous studies (e.g., Bhardwaj et al., 2011; Lopez \& Fan 2009) have
found that the enterprise's integrated supply chain contributes to rapid
internationalization. In the previous Zara case, by controlling the
entire production chain, the company uses marketing mix elements that
are standardized across the store, as well as foreign market entries in
the form of franchise stores, it has shown that it has made it possible
to have faster turnarounds (Lopez \& Fan 2009). By managing the various
stage in the supply chain by yourself, you can safely supply eligible
products at a reasonable price (Kim 2010). In addition, corporate
control across the supply chain promoted internationalization based on
the benefits of internalization of the OLI model. Managing the entire
system makes it easy to transfer organizational assets quickly and
consistently through international operations within your organization
(Zhao \& Decker 2004), like in a franchise retail store format.

\hypertarget{discussions-for-the-comparative-analysis-between-uniqlo-and-giordano}{%
\subsection{Discussions for the Comparative Analysis Between Uniqlo and
Giordano}\label{discussions-for-the-comparative-analysis-between-uniqlo-and-giordano}}

Asian apparel brands are globalized, but research on
internationalization lacked literature. The analysis of the
internationalization of Giordano and Uniqlo in this study showed a
similar pattern of entry into foreign markets. They first start in a
relatively close country, then expand to a farer country aggressively
with specifically different strategies. Giordano opened a store in a
duty-free shop and focused on expanding into the Middle East, while
UNIQLO focused on launching a global flagship store in fashion capital.
In recent years, the two companies have focused on developing countries
with high market potential. The application of the theory of
internationalization can explain the internationalization patterns of
the two Asian brands: the Uppsala model explains the second and third
patterns; while the Locational advantage of OLI model describes the
second and third patterns. In internationalization, the brand concept,
which focuses on the integrated supply chain and basic apparel of the
brand, accelerated entry into foreign markets by taking advantage of
internalization from OLI models and reducing design localization time.
This is clearly in contrast to global fast fashion retailers, which
focus on the rapid turning of fashion products, despite sharing the
commonalities of the integrated supply chain. In conclusion, it was
necessary to apply not only one theory, but also multiple theories to
explain the internationalization of Giordano and Uniqlo.

Theoretically, the findings of this study add an empirical case of Asian
brands to internationalizing literature that applies existing theories
to the case of a particular brand (e.g., children and Gin 2014; \& Lopez
Fan 2009). Previous case studies on global fashion brands such as Zara
(Lopez \& Fan 2009), H\&M, and New Look (Child \& Gin 2014) showed
similar entry patterns for brands in relatively close countries,
International lowering Uppsala model. However, regardless of
geographical and cultural proximity, the ``born-global'' movement, which
describes the aggressive expansion into foreign markets, was once again
demonstrated (Bell et al., 2001; ``Child \& Gin 2014'' depicts a
distribution scattered across markets around the world. Due to this
move, the existing Uppsala model alone does not provide a description
perfect description (Children \& Gin 2014). The case of Giordano and
Uniqlo also supported the literature that internationalization theory
alone is not enough to explain the pattern of entry into foreign
markets, but some theories like the Uppsala model and the OLI model are
necessary for their explanation international. Current research has
shown that two Asian brands return to the neighboring Asian market, as
the Uppsala model suggests, due to the potential for growth in emerging
Asian countries. This difference may be interpreted by the high brand
assets of global fast fashion brands compared to Asian retailers. Global
fast fashion brands have begun internationalization early and have more
stable positions in global markets than Asian brands, so they have more
experience, brand awareness, and brand equity. Stable brand position in
the market. As the Uppsala model explained that knowledge and experience
are an important resource for internationalization, this may now allow
us to enter more actively and diverse markets compared to Asian brands
(Forsgren 2002). If so, the question of the future is, will Asian brands
follow an ``born-global'' movement similar to global fast fashion brands
as they grow more?

In fact, the findings affect global brands of Western origin in
particular. The developing market in Asia is one of the fastest growing
markets with high potential, and the Asian brand (Giordano and UNIQLO)
was focused on this market with the benefits of empirical knowledge,
geographical and cultural proximity. Previously, when global brands
tried to enter developing countries, they were less competitive because
it was a problem competing with other brands of Western origin. But now
the brand has to compete with Asian players who have a competitive
advantage in the Asian market. To support this, there are additional
cases of Asian brands getting significant smoking cessation in Asian
countries: Korean apparel retailer ELand operates nine fashion brands in
China (Park 2007) and 182 of 2, Sales volume of more than 400 stores in
Chinese cities have increased by about 30\% each year over the past
decade (Yong 2012). Japan's unsigned two, which distributes its own
minimal design apparel and lifestyle products (Curry 2008) and appeals
to consumers, is rapidly expanding in several countries, including new
entries to eight Asian countries, 11 European countries and United.
State (Unmarked Road Website). What is the best positive factor for a
global brand with Western origin in Asia's emerging markets? Based on
the benefits of empirical knowledge in Asia, it is necessary to observe
what strategies our Competitors are pursuing in the Asian market. In
this sense, this study was an early attempt to show the current progress
of internationalization of Asian brands. Now, Western brands need to
consider how their differential competitive advantage kind can beat the
empirical knowledge of Asian competitors in emerging markets in Asia.
For example, Giordano and Uniqlo concentrated on the basic causal wear
concept by avoiding the same concept of global fast fashion.

\hypertarget{analysis-of-quick-response-based-supply-chain-management-in-uniqlo}{%
\section{Analysis of Quick Response Based Supply Chain Management in
Uniqlo}\label{analysis-of-quick-response-based-supply-chain-management-in-uniqlo}}

\hypertarget{introduction-to-the-supply-chain-management-analysis-in-uniqlo-based-on-quick-response}{%
\subsection{Introduction to the Supply Chain Management Analysis in
Uniqlo Based on Quick
Response}\label{introduction-to-the-supply-chain-management-analysis-in-uniqlo-based-on-quick-response}}

With the globalization of the global economy, competition in production
and design has intensified, and competition in the corporate supply
chain gradually shifts. The clothing industry is a traditional industry
with short cycles, various demands and other characteristics.
Diversification of customer demand, speed of market responsiveness, and
accuracy of market demand forecasts are among clothing companies facing
supply and demand uncertainty and high inventory problems. The rapid
changes in the modern clothing industry have already reduced order
times, caused more and more product varieties and excess inventory,
putting great pressure on the supply chain management of our clothing
industry. Economic growth and improved living standards make customer
demand very diversified and personalized, increasingly complex and
unpredictable. In 2014, China's 87 clothing and textile companies
accounted for 73.2 billion in inventories, 83 percent of which were
small and medium-smaller clothing companies. The situation is very
serious due to a shortage of sales channels due to overstocking and a
sharp decline in orders. Establishing a highly efficient supply chain of
customer demand, avoiding risks, and improving inventory problems will
be very realistic.

\hypertarget{key-influencing-factors-of-quick-response}{%
\subsection{Key Influencing Factors of Quick
Response}\label{key-influencing-factors-of-quick-response}}

Quick Response (QR) refers to a supply chain management method that uses
POS (POS), Electronic Data Exchange (EDI), and other information
technology to exchange information when replenishing or continuously
replenishing a particular product. To achieve common goals such as
shortening delivery cycles, reducing inventory, improving customer
service levels, and competitiveness of companies at a small number of
multiple frequencies.

\hypertarget{the-type-of-supply-chain}{%
\subsubsection{The type of supply
chain}\label{the-type-of-supply-chain}}

Depending on different classification criteria, the supply chain can be
divided into different types. Depending on the requirements of different
types of products and customers, the supply chain can be divided into
four types: functional supply chain, innovative supply chain, push
supply chain, pull supply chain, and so on. When innovative products are
in short lifecycle and faced with uncertain market demand, companies
need to respond quickly to changing market needs. The pull supply chain
is based on customer requirements. Therefore, innovative products are
best suited to use rapid response strategies to respond quickly to
customer requirements.

\hypertarget{partnerships-between-companies}{%
\subsubsection{Partnerships between
companies}\label{partnerships-between-companies}}

The partnership between each member in the supply chain, for short, the
supplier relationship is the relationship between the supplier and the
manufacturer, or the seller and the buyer. The relationship with the
supplier is a strategic partnership that shares risks and pursues common
interests between suppliers and manufacturers over a specific period of
time. After that, information sharing and unified decision-making of
each member of the supply chain greatly improved the ability of the
supply chain to respond quickly.

\hypertarget{sharing-information}{%
\subsubsection{Sharing information}\label{sharing-information}}

The supply chain is a chain of downstream companies, and if information
cannot be communicated to downstream companies in time, the ability to
respond quickly throughout the supply chain system will be greatly
reduced, causing supply chain flexibility. Each member of the supply
chain coordinates changes in the internal and external environment, and
the speed of market demand determines supply chain flexibility.
Therefore, the level of information sharing is one of the key factors
that affect suing quickly.

\hypertarget{model-abstraction-of-uniqlos-supply-chain}{%
\subsection{Model Abstraction of Uniqlo's Supply
Chain}\label{model-abstraction-of-uniqlos-supply-chain}}

The founder of UNIQLO, named Tadashi Yanai, develops and manufactures
all products under its own brand, a retail model for clothing specialty
stores, namely its own brand spa (private label apparel specialty store
retail store) The model was founded companies with integrated operations
and unified distribution. While traditional management models have so
many brokerage agents, the SPA model actually associates direct sales
and production, with 100\% full control to save costs and increase
operational efficiency. Supply.

\hypertarget{spa-model}{%
\subsubsection{SPA model}\label{spa-model}}

UNIQLO applies its own brand of advanced SPA (specialty retail store for
private label apparel) models, namely clothing specialty stores. This
model connects production and sales directly, and customers and
producers are directly connected by canceling intermediate links,
enabling them to respond to customer needs in a timely manner, as well
as consuming products at the fastest speed. The UNIQLO SPA model
consists of clothing material procurement, product planning, development
and manufacturing, distribution and retail inventory management
throughout the clothing manufacturing process. This model breaks the
traditional model of planning, production, and sales separation, while
consolidating planning, production, and sales as a whole. By canceling
traditional models of intermediaries and other links, this model
significantly reduces the cost of distribution and the cost of
production and sales. In addition, sales information for each store can
be delivered to the factory in a timely manner, and production
adjustment can be made quickly according to the sales information. The
SPA model has reduced the average inventory rotation days to 80 days and
is building a leading position in the clothing industry.

\hypertarget{the-tcmtotal-chain-management-system}{%
\subsubsection{The TCM(Total Chain Management)
system}\label{the-tcmtotal-chain-management-system}}

Uniqlo TCM system has been developed based on conventional supply chain
management (SCM). Compared to SCM, the TCM system improves the overall
capabilities of product management, focusing on the collection and
analysis of store sales data and market information, and basically
realizes centralized management and information sharing. Management and
store managers from the Marketing and Sales Planning Department
participate in weekly management meetings. Analyze all store sales data
provided by the POS system, forecast market demand, and determine
production capacity and inventory. The store simply determines the
required orders according to a specific situation and enters the
delivery date and time, and the order information is sent to the head
office. Then, the head office arranges shipping instructions, and the
distribution center arranges delivery. Each week, the head office tracks
inventory for all stores, and if the store requires additional orders,
the head office immediately transfers the relevant sales information to
the factory to check the color size and other information for the order
product. The factory then quickly adjusts with the information it
receives and receives additional orders within 10 days.

\hypertarget{main-advantages-of-uniqlos-supply-chain}{%
\subsection{Main Advantages of Uniqlo's Supply
Chain}\label{main-advantages-of-uniqlos-supply-chain}}

\hypertarget{source-control-of-the-supply-chain}{%
\subsubsection{Source control of the supply
chain}\label{source-control-of-the-supply-chain}}

The application of spa models can cancel a series of links between
agents and connect directly from production to sales to achieve the
production of raw materials from development, distribution and sales to
overall control. The entire process of supply chain management consists
of five parts: material planning, product planning, sales planning,
production planning, and sales process. Each part contains many specific
links, and the design of each particular link is very detailed and
delicate. Taking the product design process as an example, there are six
aspects: data analysis, trend analysis, sales planning development,
material planning, product development planning, and design. The source
control of UNIQLO's supply chain is very important for material
development and quality. Material determination is determined by a group
member consisting of a single person, not a designer, a product planning
staff, materials and people in product development. After the
discussion, the president makes a final decision to ensure the quality
of the material. Uniqlo has also established a long-term strategic
cooperation partnership to secure a source of raw materials. Therefore,
when market demand changes, raw materials can be supplied in a timely
manner, avoiding the risk of market changes.

\hypertarget{win-win-model}{%
\subsubsection{Win-win model}\label{win-win-model}}

There are two types of UNIQLO production models produced at our own
factory, and the other is production outsourcing. Uniqlo reduces
production costs and improves efficiency through outsourcing production.
At the same time, you will face the problem of choosing a manufacturer.
To effectively solve this problem, we set strict screening criteria for
selecting about 70 manufacturers in more than 100 manufacturers. UNIQLO
also has established a production management company that tightly
controls the quality and production of its products. In addition, the
craftsman system is applied to UNIQLO. The craftsman system refers to
the technical guidance of a group of craftsmen sent from the head
office. The Artisan Group consists of 30 technical staff with decades of
production experience, each of whom is responsible for coaching 5-10
outsourcing companies. These experienced craftsmen provide technical
guidance at outsourced clothing factories over the long term to ensure
that the products of other manufacturers and their factories do not
differ in terms of quality.

\hypertarget{sound-operation-based-on-information-analysis}{%
\subsubsection{Sound operation based on information
analysis}\label{sound-operation-based-on-information-analysis}}

Uniqlo's core competitiveness is its rapid supply chain, and its strong
ability to analyze information is key to the rapid response of the
supply chain. Unlike traditional operational mechanisms, UNIQLO's store
operations are directly managed by the head office. Through information
systems, the head office tracks inventory management for each store on a
daily basis, so it collects all the sales data for each store to form a
vast database of internal information integration. Forecast and
inventory management based on the analysis of the data that companies
can sell.

\hypertarget{inspirations-to-the-apparel-industry}{%
\subsection{Inspirations to the Apparel
Industry}\label{inspirations-to-the-apparel-industry}}

\hypertarget{establishment-of-qr-information-systems}{%
\subsubsection{Establishment of QR information
systems}\label{establishment-of-qr-information-systems}}

The ability of supply chain information systems to respond at high speed
plays an important role in the enterprise. Companies can promote
collaboration between members to improve the level of supply chain
information sharing through information technology and to improve
operational efficiency in the supply chain. Without information system
support, UNIQLO will significantly reduce the rate of collection and
analysis of sales data and not be able to respond quickly to market
changes. Compared to UNIQLO's SPA models and TCM systems, many apparel
companies today generally pay less attention to the construction of
information systems and information technology that have responded to
customer needs at a lower level and caused a lack of supply chain.
Flexibility. When clothing companies are trying to establish a fast
supply chain like UNIQLO, they can build a perfect information system
combined with their business, improve supply chain information sharing
levels, reduce lead times, apply POS systems, Improve industry chains
that meet the unique competitiveness of the entire supply chain, such as
rapid improvement of sales data, market information collection and
analysis, and centralized management and information sharing, further
improving the operational efficiency and economic benefits of companies.

\hypertarget{improve-inventory-management-levels}{%
\subsubsection{Improve inventory management
levels}\label{improve-inventory-management-levels}}

UNIQLO abandoned its traditional inventory management model through
weekly TCM business meetings. During this meeting, the management group
analyzes sales data, forecasts market demand, and determines production
capacity and inventory. Based on this information system, UNIQLO can
track unit inventory products daily and determine product activities
such as additional orders, discount sales, and inventory clearing. With
effective inventory management, UNIQLO can solve corporate problems to
meet market demands, reduce inventory costs, and respond to market
changes. Superior inventory management reduces inventory costs and
improves the operational efficiency of supply chain inventory
management. Due to inadequate management, most clothing companies need
to face overstock ingress problems, along with increased inventory costs
and a significant decline in supply chain capacity. Clothing companies
need to learn about the entire supply chain management system from
UNIQLO, promote cooperation between upstream and downstream companies in
the supply chain, and improve the level of inventory management and
information levels. Share. Therefore, clothing companies should pay
attention to the effectiveness of information in the process of
information transfer and timeliness in the process of establishing a
marketing channel. At the same time, you should consider how to avoid
high-quality inventory risks and make planned promotions in sales
terminals.

\hypertarget{strengthen-supplier-management}{%
\subsubsection{Strengthen supplier
management}\label{strengthen-supplier-management}}

Supplier management is a very important link to supply chain management.
Partnerships with suppliers are beneficial to companies to improve the
overall ability to respond in the supply chain and reduce manufacturing
costs. UNIQLO has developed strict selection criteria for supplier
selection and evaluation on consistency between aspects of supplier
production level screening standards and unique goals. Uniqlo's TCM
model is a typical example. Under this system, we have established a
production management company that strictly controls production and
quality. In addition, a group of artisans was sent to the manufacturer's
factory to lead long-term production with selected suppliers already
forming Win-Win models. This not only reduces production costs, but also
provides effective control of production and quality. In comparison with
UNIQLO Supplier Management, there are many issues for other clothing
companies, including explicit screening standards for supplier selection
and lack of awareness of cooperation. To establish a rapid response
supply chain, you need to make the most of UNIQLO's supplier management
practices, including establishing scientific metrics, choosing the best
strategic partner, and establishing supplier quality.

\hypertarget{conclusion}{%
\section{Conclusion}\label{conclusion}}

From the analysis above, we know that for apparel industry, the supply
chain management play an important role in making the company obtain
more profit. As we have analyzed Uniqlo as our analysis case, we find
out that Uniqlo adopts the SPA model and the Total Chain Management
system to make its supply chain run in a very high efficiency. What's
more, the Quick Response concept developed by Uniqlo very well serve the
reforming process of Uniqlo's supply chain, as well as its attention
paid to source control, win-win model and sound operation. In
conclusion, we can get some inspiration from Uniqlo's supply chain
model. That is, a corporate needs to establishes its Quick Response
information system for better reformation of supply chain. Also, it
needs to improve the inventory management levels. Besides these, another
key influencing factor is that it needs to strengthen the supplier
management system. Other companies can learn from these major points to
enhance the supply chain system and the Uniqlo company, as well, can
improve its supply chain system from the aspects above.

\pagebreak

\hypertarget{about-the-author}{%
\section*{About the author}\label{about-the-author}}
\addcontentsline{toc}{section}{About the author}

\pagebreak

\hypertarget{bibliography}{%
\section*{Bibliography}\label{bibliography}}
\addcontentsline{toc}{section}{Bibliography}

\hypertarget{refs}{}
\leavevmode\hypertarget{ref-innovative4}{}%
Agarwal, Sanjeev, and Sridhar N. Ramaswami. 1992. ``Choice of Foreign
Market Entry Mode: Impact of Ownership, Location and Internalization
Factors.'' \emph{Journal of International Business Studies} 23 (1).
Palgrave Macmillan UK: 1--27.
\url{https://doi.org/10.1057/palgrave.jibs.8490257}.

\leavevmode\hypertarget{ref-innovative5}{}%
Andersen, Otto, and Low Suat Kheam. 1998. ``Resource-Based Theory and
International Growth Strategies: An Exploratory Study.''
\emph{International Business Review} 7 (2). Elsevier B.V.: 163--84.
\url{https://doi.org/10.1016/S0969-5931(98)00004-3}.

\leavevmode\hypertarget{ref-innovative1}{}%
Andersson, Svante. 2000. ``The Internationalization of the Firm from an
Entrepreneurial Perspective.'' \emph{International Studies of Management
\& Organization} 30 (1). Taylor \& Francis, Ltd.: 63--92.
\url{https://doi.org/10.1080/00208825.2000.11656783}.

\leavevmode\hypertarget{ref-innovativeInternationalisation}{}%
Andersson, Svante, and Ingemar Wictor. 2003. ``Innovative
Internationalisation in New Firms: Born Globals--the Swedish Case.''
\emph{Journal of International Entrepreneurship} 11 (1). Kluwer Academic
Publishers: 249--75. \url{https://doi.org/10.1023/A:1024110806241}.

\leavevmode\hypertarget{ref-innovative7}{}%
Childs, Michelle Lynn, and Byoungho Jin. 2014. ``Is Uppsala Model Valid
to Fashion Retailers? An Analysis from Internationalisation Patterns of
Fast Fashion Retailers.'' \emph{Journal of Fashion Marketing and
Management} 18 (16). Emerald Publishing Limited: 36--51.
\url{https://doi.org/10.1108/JFMM-10-2012-0061}.

\leavevmode\hypertarget{ref-innovative2}{}%
Cox, Andrew. 1999. ``Power, Value and Supply Chain Management.''
\emph{Supply Chain Management} 4 (4). MCB UP Ltd: 167--75.
\url{https://doi.org/10.1108/13598549910284480}.

\leavevmode\hypertarget{ref-innovative8}{}%
Dibb, Sally. 1996. ``The Impact of the Changing Marketing Environment in
the Pacific Rim: Four Case Studies.'' \emph{International Journal of
Retail \& Distribution Management} 24 (1). MCB University Press: 16--29.
\url{https://doi.org/10.1108/09590559610131691}.

\leavevmode\hypertarget{ref-innovative9}{}%
Dunning, John H. 1977. ``Trade, Location of Economic Activity and the
MNE: A Search for an Eclectic Approach.'' \emph{The International
Allocation of Economic Activity} 30 (1). Palgrave Macmillan, London:
395--418. \url{https://doi.org/10.1007/978-1-349-03196-2_38}.

\leavevmode\hypertarget{ref-innovative10}{}%
---------. 1988. ``The Eclectic Paradigm of International Production: A
Restatement and Some Possible Extensions.'' \emph{Journal of
International Business Studies} 19 (1). Palgrave Macmillan UK: 1--31.
\url{https://doi.org/10.1057/palgrave.jibs.8490372}.

\leavevmode\hypertarget{ref-innovative3}{}%
Edition Department of Hanbai Kakushin. 2000. \emph{ABC Kaikaku No Zenbou
(the Full Picture of All Better Change Activity)}. \emph{Hnbai
Kakushin}.

\leavevmode\hypertarget{ref-innovative11}{}%
Forsgren, Mats. 2015. ``The Concept of Learning in the Uppsala
Internationalization Process Model: A Critical Review.''
\emph{Knowledge, Networks and Power} 19 (1). Palgrave Macmillan, London:
88--110. \url{https://doi.org/10.1057/9781137508829_4}.

\leavevmode\hypertarget{ref-innovative12}{}%
Hayes, S.G., and Nicola Jones. 2006. ``Fast Fashion: A Financial
Snapshot.'' \emph{Journal of Fashion Marketing and Management} 10 (3).
Emerald Group Publishing Limited: 282--300.
\url{https://doi.org/10.1108/13612020610679277}.

\leavevmode\hypertarget{ref-innovative13}{}%
Jackson, Paul, and Leigh Sparks. 2005. ``Retail Internationalisation:
Marks and Spencer in Hong Kong.'' \emph{International Journal of Retail
\& Distribution Management} 33 (10). Emerald Group Publishing Limited:
766--83. \url{https://doi.org/10.1108/09590550510622308}.

\leavevmode\hypertarget{ref-innovative14}{}%
Johanson, Jan, and Jan-Erik Vahlne. 1977. ``The Internationalization
Process of the Firm --- A Model of Knowledge Development and Increasing
Foreign Market Commitments.'' \emph{Journal of International Business
Studies} 10 (3). Palgrave Macmillan, London: 23--32.
\url{https://doi.org/10.1057/palgrave.jibs.8490676}.

\leavevmode\hypertarget{ref-innovative16}{}%
Lopez, Carmen, and Ying Fan. 2009. ``Internationalisation of the Spanish
Fashion Brand Zara.'' \emph{Journal of Fashion Marketing and Management}
13 (2). Emerald Group Publishing Limited: 279--96.
\url{https://doi.org/10.1108/13612020910957770}.

\leavevmode\hypertarget{ref-innovative15}{}%
Saxena, Ravindra P., and Pradeep K. Khandelwal. 2010. ``Is the Magic of
`Feel Good' and `Look Great' at Giordano Still Working?''
\emph{Management Decision} 48 (3). Emerald Group Publishing Limited:
440--55. \url{https://doi.org/10.1108/00251741011037792}.

\leavevmode\hypertarget{ref-innovative6}{}%
Vertica, Bhardwaj, Megan, Eickman, and Rodney. 2011. ``A Case Study on
the Internationalization Process of a `Born-Global' Fashion Retailer.''
\emph{International Review of Retail, Distribution and Consumer
Research}, January. Taylor \& Francis (Routledge).

\end{document}
